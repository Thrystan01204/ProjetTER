\documentclass[a4paper]{report}
\usepackage{a4wide}
\usepackage[utf8]{inputenc}
\usepackage[T1]{fontenc}
\usepackage[french]{babel}
\usepackage[babel=true]{csquotes} % guillemets français
\usepackage{graphicx}
\usepackage{float}

\author{SALVAN Corentin - MEHONG-SHIT-LI Matthieu - MARTINEZ Damien \\ BAIRY ONAPA Mikaël - LE LIDEC Tristan}
\title{\underline{Rapport Projet TER : Conception d'une surveillance vidéo médicale}}

\begin{document}
    \maketitle
    \chapter{Introduction}
    \section{Description}
    Le but de ce projet est de réaliser une surveillance vidéo avec deux composantes, la détection de chute et la reconnaissance faciale.
    Pour la détection de chute, lorsqu'une chute sera détecté une alarme sera activé afin d'alerter le personnel soignant. Depuis une interface graphique l'utilisateur pourra visualiser en temps réel le flux vidéo, choisir la caméra, mettre en pause le flux vidéo et désactiver l'alarme. Concernant la reconnaissance faciale, l'utilisateur connaitra le nom et le prénom de la personne si il est enregistré dans la base de données de l'établissement, ce qui permettra de détecter les éventuels intrus.
    \section{Plateforme}

    Sur la partie matériel nous avons choisi d'utiliser un Rasberry Pi, du fait de son aspect compact et de sa grande capacité de calcul. Concernant la partie logiciel, nous nous sommes tourné vers le langage de programmation Python car c'est un langage polyvalent avec une grande communauté et possède plusieurs librairies utiles à notre projet.

    \section{Part2}
    \section{Part3}
\end{document}